%&clatex/templateAll
\endofdump

%\documentclass[11pt]{article}

%\usepackage{template}
%\useTemplateStatic[a]

%\usepackage{src/template}

%! Generated with commit hash: "2d577c1673fa5e6ab29b67ee03427398457bea80"
%! sha1sum of templateAll.fmt: "c21f92af7c81cf9d5f7bc68ab2fa7446e9198660"

\date{2022 -- 02 -- 28}
\title{Making an operating system on bare-metal ARM}
\useTemplate[english]

\usepackage[outputdir=out]{minted}

\def\shell#1{\mintinline{shell}{#1}}


\begin{document}
    \maketitle

    \section{Installing the toolchain}

    First of all I am assuming you are running linux.
    If you are feeling lucky, you can also try this on any posix-like environment, but beware that things might not work.

    If you are developing under windows you can use \href{https://docs.microsoft.com/en-us/windows/wsl/install}{WSL}.
    Bare in mind that things also might not work as expected.

    With this disclaimer out of the way we can start the first steps in compiling a program for ARM!

    \subsection{Installing the compiler}

    Installing the compiler should be as easy as running the command \shell{pacman -S arm-none-eabi-gcc}.
    The package is even available under Debian, so your package manager \textit{should} have it.

    If you want to compile it manually you have to execute the following commands:

    \begin{minted}{shell}

    \end{minted}

    \subsection{Installing a debugger}

    \subsection{Installing QEMU}

    \subsection{Installing mkimage and UBoot}


    \section{\shell{make}-ing first contact}



\end{document}


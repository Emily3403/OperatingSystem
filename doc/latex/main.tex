%&clatex/templateAll
\endofdump

%\documentclass[11pt]{article}

%\usepackage{template}
%\useTemplateStatic[a]

%\usepackage{src/template}

%! Generated with commit hash: "2d577c1673fa5e6ab29b67ee03427398457bea80"
%! sha1sum of templateAll.fmt: "c21f92af7c81cf9d5f7bc68ab2fa7446e9198660"

\date{2022 -- 02 -- 28}
\title{Making an operating system on a Raspberry PI 3b with C and Zig}
\useTemplate[english]

\usepackage[outputdir=out]{minted}

\def\shell#1{\mintinline{shell}{#1}}
\def\sh#1{\mintinline{shell}{#1}}
\def\zig{\sh{zig}}



\begin{document}
    \maketitle


    \section{Installing the toolchain}

    First of all I am assuming you are running linux.
    If you are feeling lucky, you can also try this on any posix-like environment, but beware that things might not work.

    If you are developing under windows you can use \href{https://docs.microsoft.com/en-us/windows/wsl/install}{WSL}.
    Bare in mind that things also might not work as expected.

    With this disclaimer out of the way we can start the first steps in compiling a program for ARM!

    \subsection{Choosing a compiler}

    \subsubsection{GCC}

    Unfortunately you cannot use the standard \shell{gcc} to compile your Operating system. If you are running any sort of desktop PC, the CPU architecture of the Raspberry 3b is
    different from that you are running on. To check it run the command \sh{gcc -dumpmachine}. I get \sh{x86_64-pc-linux-gnu}, yours is probably similar.

    The toolchain we need is the \sh{gcc-arm-none-eabi}. You can get your own by executing one of the following
    % TODO: Test if this actually works
    \begin{itemize}
        \item \sh{sudo pacman -S arm-none-eabi-gcc}
        \item \sh{sudo apt install gcc-arm-none-eabi}
        \item \sh{sudo dnf install arm-none-eabi-gcc}
    \end{itemize}

    You can also download the \href{https://developer.arm.com/tools-and-software/open-source-software/developer-tools/gnu-toolchain/gnu-rm/downloads}{official toolchain} from ARM.

    With \sh{gcc} you should have everything up and running to compile your first ARM program.

    \subsubsection{Zig}

    \zig{} is a programming language and compiler, which aims at replacing C\@.

    The \zig{} compiler ships with everything needed to compile for all target architectures and even libc and musl.
    This provides the benefit of easily migrating to a different CPU version or even architecture.

    In the linked git repository you will find a up-to-date version of the zig compiler, which is used in our toolchain.
    This may or may not be better for your usecase.

    \subsubsection{Clang}

    % TODO: Document clang
    Clang is currently untested and undocumented.

    \subsection{Installing a debugger}

    \subsection{Installing QEMU}

    \subsection{Installing mkimage and UBoot}


    \section{First contact}


    \newpage

\begin{tikzpicture}[node distance = 5em]
    \node (1) {$\{X,\neg Y\}$};
    \node (3) [below right of=1] {$\{X,Z\}$};
    \node (2) [above right of=3] {$\{Y,Z\}$};
    \draw[bend left] (1) to (3);
    \draw[bend right=40] (2) to (3);
\end{tikzpicture}


\end{document}

